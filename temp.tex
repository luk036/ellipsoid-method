\documentclass[]{article}
\usepackage{lmodern}
\usepackage{amssymb,amsmath}
\usepackage{ifxetex,ifluatex}
\usepackage{fixltx2e} % provides \textsubscript
\ifnum 0\ifxetex 1\fi\ifluatex 1\fi=0 % if pdftex
  \usepackage[T1]{fontenc}
  \usepackage[utf8]{inputenc}
\else % if luatex or xelatex
  \ifxetex
    \usepackage{mathspec}
  \else
    \usepackage{fontspec}
  \fi
  \defaultfontfeatures{Ligatures=TeX,Scale=MatchLowercase}
\fi
% use upquote if available, for straight quotes in verbatim environments
\IfFileExists{upquote.sty}{\usepackage{upquote}}{}
% use microtype if available
\IfFileExists{microtype.sty}{%
\usepackage{microtype}
\UseMicrotypeSet[protrusion]{basicmath} % disable protrusion for tt fonts
}{}
\usepackage[unicode=true]{hyperref}
\hypersetup{
            pdftitle={USING SIAM'S LaTeX~MACROS{[}\^{}2{]}},
            pdfauthor={TeX Production{[}\^{}1{]}},
            pdfborder={0 0 0},
            breaklinks=true}
\urlstyle{same}  % don't use monospace font for urls
\usepackage{longtable,booktabs}
\IfFileExists{parskip.sty}{%
\usepackage{parskip}
}{% else
\setlength{\parindent}{0pt}
\setlength{\parskip}{6pt plus 2pt minus 1pt}
}
\setlength{\emergencystretch}{3em}  % prevent overfull lines
\providecommand{\tightlist}{%
  \setlength{\itemsep}{0pt}\setlength{\parskip}{0pt}}
\setcounter{secnumdepth}{0}
% Redefines (sub)paragraphs to behave more like sections
\ifx\paragraph\undefined\else
\let\oldparagraph\paragraph
\renewcommand{\paragraph}[1]{\oldparagraph{#1}\mbox{}}
\fi
\ifx\subparagraph\undefined\else
\let\oldsubparagraph\subparagraph
\renewcommand{\subparagraph}[1]{\oldsubparagraph{#1}\mbox{}}
\fi

\title{USING SIAM'S LaTeX~MACROS{[}\^{}2{]}}
\author{TeX Production{[}\^{}1{]}}
\date{}

\begin{document}
\maketitle
\begin{abstract}
Documentation is given for use of the SIAM LaTeX~macros. These macros
are now compatible with LaTeX\(2_{\varepsilon}\). Instructions and
suggestions for compliance with SIAM style standards are also included.
Familiarity with standard LaTeX~commands is assumed.
\end{abstract}

\section{Introduction}\label{introduction}

This file is documentation for the SIAM LaTeX~macros, and provides
instruction for submission of your files (test).

To accommodate authors who electronically typeset their manuscripts,
SIAM supports the use of LaTeX. To ensure quality typesetting according
to SIAM style standards, SIAM provides a LaTeX~macro style file. Using
LaTeX~to format a manuscript should simplify the editorial process and
lessen the author's proofreading burden. However, it is still necessary
to proofread the galley proofs with care.

Electronic files should not be submitted until the paper has been
accepted, and then not until requested to do so by someone in the SIAM
office. Once an article is slated for an issue, someone from the SIAM
office will contact the author about any or all of the following:
editorial and stylistic queries, supplying the source files (and any
supplementary macros) for the properly formatted article, and handling
figures.

When submitting electronic files (electronic submissions) (to
{tex@siam.org}) include the journal, issue, and author's name in the
subject line of the message. Authors are responsible for ensuring that
the paper generated from the source files exactly matches the paper that
was accepted for publication by the review editor. If it does not,
information on how it differs should be indicated in the transmission of
the file. When submitting a file, please be sure to include any
additional macros (other than those provided by SIAM) that will be
needed to run the paper.

SIAM uses MS-DOS-based computers for LaTeX~processing. Therefore all
filenames should be restricted to eight characters or less, plus a three
character extension.

Once the files are corrected here at SIAM, we will mail the revised
proofs to be read against the original edited hardcopy manuscript. We
are not set up to shuttle back and forth varying electronic versions of
each paper, so we must rely on hard copy of the galleys. The author's
proofreading is an important but easily overlooked step. Even if SIAM
were not to introduce a single editorial change into your manuscript,
there would still be a need to check, because electronic transmission
can introduce errors.

The distribution contains the following items: {siamltex.cls}, the main
macro package based on {article.cls}; {siam10.clo}, for the ten-point
size option;, a style option for equation numbering (see §3 for an
explanation); and {siam.bst}, the style file for use with {Bib}TeX. Also
included are this file {docultex.tex} and a sample file {lexample.tex}.
The sample file represents a standard application of the macros. The
rest of this paper will highlight some keys to effective macro use, as
well as point out options and special cases, and describe SIAM style
standards to which authors should conform.

\section{Headings}\label{headings}

The top matter of a journal paper falls into a standard format. It
begins of course with the \texttt{\textbackslash{}documentclass} command

\begin{verbatim}
\documentclass{siamltex}
\end{verbatim}

Other class options can be included in the bracketed argument of the
command, separated by commas.

Optional arguments include:

\begin{description}
\item[final]
Without this option, lines which extend past the margin will have black
boxes next to them to help authors identify lines that they need to fix,
by re-writing or inserting breaks. \texttt{final} turns these boxes off,
so that very small margin breaks which are not noticible will not cause
boxes to be generated.
\item[oneeqnum]
Normally \texttt{siamltex.cls} numbers equations, tables, figures, and
theorem environments with a decimal number, composed of the section of
the paper, a period, and the number of the enumerated object (example:
1.2). The sequence of numbering is also restarted with each new section,
so that, for example, the last equation of section 3 may be 3.10, but
the first equation of section 4 would be 4.1. Using \texttt{oneeqnum}
numbers all equations consecutively throughout a paper with a single
digit.
\item[onethmnum]
Using \texttt{onethmnum} numbers all theorem-like environments
consecutively throughout a paper with a single digit.
\item[onefignum]
Using \texttt{onethmnum} numbers all figures consecutively throughout a
paper with a single digit.
\item[onetabnum]
Using \texttt{onethmnum} numbers all tables consecutively throughout a
paper with a single digit.
\end{description}

The title and author parts are formatted using the
\texttt{\textbackslash{}title} and \texttt{\textbackslash{}author}
commands as described in Lamport~{[}@Lamport{]}. The
\texttt{\textbackslash{}date} command is not used.
\texttt{\textbackslash{}maketitle} produces the actual output of the
commands.

The addresses and support acknowledgments are put into the
\texttt{\textbackslash{}author} commands via
\texttt{\textbackslash{}thanks}. If support is overall for the authors,
the support acknowledgment should be put in a
\texttt{\textbackslash{}thanks} command in the
\texttt{\textbackslash{}title}. Specific support should go following the
addresses of the individual authors in the same
\texttt{\textbackslash{}thanks} command.

Sometimes authors have support or addresses in common which necessitates
having multiple \texttt{\textbackslash{}thanks} commands for each
author. Unfortunately LaTeX~does not normally allow this, so a special
procedure must be used. An example of this procedure follows. Grant
information can also be run into both authors' footnotes.

\begin{verbatim}
\title{TITLE OF PAPER}

\author{A.~U. Thorone\footnotemark[2]\ \footnotemark[5]
\and A.~U. Thortwo\footnotemark[3]\ \footnotemark[5]
\and A.~U. Thorthree\footnotemark[4]}

\begin{document}
\maketitle

\renewcommand{\thefootnote}{\fnsymbol{footnote}}

\footnotetext[2]{Address of A.~U. Thorone}
\footnotetext[3]{Address of A.~U. Thortwo}
\footnotetext[4]{Address of A.~U. Thorthree}
\footnotetext[5]{Support in common for the first and second
authors.}

\renewcommand{\thefootnote}{\arabic{footnote}}
\end{verbatim}

Notice that the footnote marks begin with because the first mark (the
asterisk) will be used in the title for date-received information by
SIAM, even if not already used for support data. This is just one
example; other situations follow a similar pattern.

Following the author and title is the abstract, key words listing, and
AMS subject classification number (s), designated using the
\texttt{\{abstract\}}, \texttt{\{keywords\}}, and \texttt{\{AMS\}}
environments. If there is only one AMS number, the commands
\texttt{\textbackslash{}begin\{AM\}} and
\texttt{\textbackslash{}end\{AM\}} are used instead of \texttt{\{AMS\}}.
This causes the heading to be in the singular. Authors are responsible
for providing AMS numbers. They can be found in the Annual Index of Math
Reviews, or through {e-Math} ({telnet e-math.ams.com}; login and
password are both {e-math}).

Left and right running heads should be provided in the following way.

\begin{verbatim}
\pagestyle{myheadings}
\thispagestyle{plain}
\markboth{A.~U. THORONE AND A.~U. THORTWO}{SHORTER PAPER
TITLE} 
\end{verbatim}

\section{Equations and mathematics}\label{equations-and-mathematics}

One advantage of LaTeX~is that it can automatically number equations and
refer to these equation numbers in text. While plain TeX's method of
equation numbering (explicit numbering using
\texttt{\textbackslash{}leqno}) works in the SIAM macro, it is not
preferred except in certain cases. SIAM style guidelines call for
aligned equations in many circumstances, and LaTeX's
\texttt{\{eqnarray\}} environment is not compatible with
\texttt{\textbackslash{}leqno} and LaTeX~is not compatible with the
plain TeX~command \texttt{\textbackslash{}eqalign} and
\texttt{\textbackslash{}leqalignno}. Since SIAM may have to alter or
realign certain groups of equations, it is necessary to use the
LaTeX~system of automatic numbering.

Sometimes it is desirable to designate subequations of a larger equation
number. The subequations are designated with (roman font) letters
appended after the number. SIAM has supplemented its macros with the
{subeqn.clo} option which defines the environment
\texttt{\{subequations\}}.

\begin{verbatim}
\begin{subequations}\label{EKx}
\begin{equation}
 y_k =  B  y_{k-1} +  f, \qquad k=1,2,3,\ldots  
\end{equation}
for  any initial vector \( y_0\).   Then 
\begin{equation}
 y_k\rightarrow  u \mbox{\quad iff\quad} \rho( B)<1.
\end{equation}
\end{subequations}
\end{verbatim}

All equations within the \texttt{\{subequations\}} environment will keep
the same overall number, but the letter designation will increase.

Clear equation formatting using TeX~can be challenging. Aside from the
regular TeX~documentation, authors will find Nicholas J. Higham's book
{\emph{Handbook of Writing for the Mathematical Sciences}}~{[}@Higham{]}
useful for guidelines and tips on formatting with TeX. The book covers
many other topics related to article writing as well.

Authors commonly make mistakes by using \texttt{\textless{}},
\texttt{\textgreater{}}, \texttt{\textbackslash{}mid}, and
\texttt{\textbackslash{}parallel} as delimiters, instead of
\texttt{\textbackslash{}langle}, \texttt{\textbackslash{}rangle},
\texttt{\textbar{}}, and \texttt{\textbackslash{}\textbar{}}. The
incorrect symbols have particular meanings distinct from the correct
ones and should not be confused.

\begin{longtable}[]{@{}llll@{}}
\caption{Illustration of incorrect delimiter use.}\tabularnewline
\toprule
\texttt{\textless{}x,\ y\textgreater{}} & \(<x, y>\) &
\texttt{\textbackslash{}langle\ x,\ y\textbackslash{}rangle} &
\$\langle x, y\rangle \$\tabularnewline
\texttt{5\ \textless{}\ \textbackslash{}mid\ A\ \textbackslash{}mid} &
\$5 \textless{} \mid A \mid \$ &
\texttt{5\ \textless{}\ \textbar{}A\textbar{}} & \$5 \textless{}
\textbar{}A\textbar{} \$\tabularnewline
\texttt{6x\ =\ \textbackslash{}parallel\ x} & & &\tabularnewline
\texttt{-\ 1\textbackslash{}parallel\_\{i\}} &
\(6x = \parallel x - 1\parallel_{i}\) &
\texttt{6x\ =\ \textbackslash{}\textbar{}x\ -\ 1\textbackslash{}\textbar{}\_\{i\}}
& \(6x = \| x - 1\|_{i}\)\tabularnewline
\bottomrule
\end{longtable}

Another common author error is to put large (and even medium sized)
matrices in-line with the text, rather than displaying them. This
creates unattractive line spacing problems, and should be assiduously
avoided. Text-sized matrices (like \(({a \atop b} {b \atop c})\)) might
be used but anything much more complex than the example cited will not
be easy to read and should be displayed.

More information on the formatting of equations and aligned equations is
found in Lamport~{[}@Lamport{]}. Authors bear primary responsibility for
formatting their equations within margins and in an aesthetically
pleasing and informative manner.

The SIAM macros include additional roman math words, or ``log-like''
functions, to those provided in standard TeX. The following commands are
added: \texttt{\textbackslash{}const}, \texttt{\textbackslash{}diag},
\texttt{\textbackslash{}grad}, \texttt{\textbackslash{}Range},
\texttt{\textbackslash{}rank}, and \texttt{\textbackslash{}supp}. These
commands produce the same word as the command name in math mode, in
upright type.

\section{Special fonts}\label{special-fonts}

SIAM supports the use of the AMS-TeX~fonts (version 2.0 and later). The
package \texttt{amsfonts} can be included with the
command\texttt{\textbackslash{}usepackage\{amsfonts\}}. This package is
part of the AMS-LaTeX~distribution, available from the AMS or from the
Comprehensive TeX Archive Network (anonymous ftp to ftp.shsu.edu). The
blackboard bold font in this font package can be used for designating
number sets. This is preferable to other methods of combining letters
(such as I and R for the real numbers) to produce pseudo-bold letters
but this is tolerable as well. Typographically speaking, number sets may
simply be designated using regular bold letters; the blackboard bold
typeface was designed to fulfil a desire to simulate the limitations of
a chalk board in printed type.

\subsection{Punctuation}\label{punctuation}

All standard punctuation and all numerals should be set in roman type
(upright) even within italic text. The only exceptions are periods and
commas. They may be set to match the surrounding text.

References to sections should use the symbol §, generated by
\texttt{\textbackslash{}S}. (If the reference begins a sentence, the
term ``Section'' should be spelled out in full.) Authors should not
redefine \texttt{\textbackslash{}S}, say, to be a calligraphic S,
because \texttt{\textbackslash{}S} must be reserved for use as the
section symbol.

Authors sometimes confuse the use of various types of dashes. Hyphens
(\texttt{-}, -) are used for some compound words (many such words should
have no hyphen but must be run together, like ``nonzero,'' or split
apart, like ``well defined''). Minus signs (\texttt{\$-\$}, \(-\))
should be used in math to represent subtraction or negative numbers. En
dashes (\texttt{-\/-}, --) are used for ranges (like 3--5,
June--August), or for joined names (like Runge--Kutta). Em dashes
(\texttt{-\/-\/-}, ---) are used to set off a clause---such as this
one---from the rest of the sentence.

\subsection{Text formatting}\label{text-formatting}

SIAM style preferences do not make regular use of the
\texttt{\{enumerate\}} and \texttt{\{itemize\}} environments. Instead,
{siamltex.cls} includes definitions of two alternate list environments,
\texttt{\{remunerate\}} and \texttt{\{romannum\}}. Unlike the standard
itemized lists, these environments do not indent the secondary lines of
text. The labels, whether defaults or the optional user-defined, are
always aligned flush right.

The \texttt{\{remunerate\}} environment consecutively numbers each item
with an arabic numeral followed by a period. This number is always
upright, even in slanted environments. (For those wondering at the
unusual naming of this environment, it comes from Seroul and
Levy's~{[}@SerLev{]} definition of a similar macro for plain TeX:
\texttt{\textbackslash{}meti} which is \texttt{\textbackslash{}item}
spelled backwards. Thus \texttt{\{remunerate\}} a portion of
\texttt{\{enumerate\}} spelled backwards.)

The \texttt{\{romannum\}} environment consecutively numbers each item
with a lower-case roman numeral enclosed in parentheses. This number
will always be upright within slanted environments (as in theorems).

\section{Theorems and Lemmas}\label{theorems-and-lemmas}

Theorems, lemmas, corollaries, definitions, and propositions are covered
in the SIAM macros by the theorem-environments \texttt{\{theorem\}},
\texttt{\{lemma\}}, \texttt{\{corollary\}}, \texttt{\{definition\}} and
\texttt{\{proposition\}}. These are all numbered in the same sequence
and produce labels in small caps with an italic body. Other environments
may be specified by the \texttt{\textbackslash{}newtheorem} command.
SIAM's style is for Remarks and Examples to appear with italic labels
and an upright roman body.

\begin{verbatim}
\begin{theorem}
Sample theorem included for illustration.  
Numbers and parentheses, like equation \((3.2)\), should be set 
in roman type.  Note that words (as opposed to ``log-like''
functions) in displayed equations, such as
$\) x^2 = Y^2 \sin z^2 \mbox{ for all } x \($
will appear in italic type in a theorem, though normally
they should appear in roman.\end{theorem}
\end{verbatim}

This sample produces Theorem 4.1 below.

Sample theorem included for illustration. Numbers and parentheses, like
equation \((3.2)\), should be set in roman type. Note that words (as
opposed to ``log-like'' functions) in displayed equations, such as
\[x^2 = Y^2 \sin z^2 \mbox{ for all } x\] will appear in italic type in
a theorem, though normally they should appear in roman.

Proofs are handled with the \texttt{\textbackslash{}begin\{proof\}}
\texttt{\textbackslash{}end\{proof\}} environment. A ``QED'' box ~is
created automatically by \texttt{\textbackslash{}end\{proof\}}, but this
should be preceded with a \texttt{\textbackslash{}qquad}.

Named proofs, if used, must be done independently by the authors. SIAM
style specifies that proofs which end with displayed equations should
have the QED box two ems (\texttt{\textbackslash{}qquad}) from the end
of the equation on line with it horizontally. Below is an example of how
this can be done:

\begin{verbatim}
{\em Proof}. Proof of the previous theorem 
                .
                .
                .
thus,
\[
a^2 + b^2 = c^2 \qquad\endproof
\]
\end{verbatim}

\section{Figures and tables}\label{figures-and-tables}

Figures and tables sometimes require special consideration. Tables in
SIAM style are need to be set in eight point size by using the
\texttt{\textbackslash{}footnotesize} command inside the
\texttt{\textbackslash{}begin\{table\}} environment. Also, they should
be designed so that they do not extend beyond the text margins.

SIAM style requires that no figures or tables appear in the references
section of the paper. LaTeX~is notorious for making figure placement
difficult, so it is important to pay particular attention to figure
placement near the references in the text. All figures and tables should
be referred to in the text.

SIAM supports the use of {epsfig} for including {PostScript} figures.
All {PostScript} figures should be sent in separate files. See the
{epsfig} documentation (available via anonymous ftp from CTAN:
ftp.shsu.edu) for more details on the use of this style option. It is a
good idea to submit high-quality hardcopy of all {PostScript} figures
just in case there is difficulty in the reproduction of the figure.
Figures produced by other non-TeX~methods should be included as
high-quality hardcopy when the manuscript is submitted.

{PostScript} figures that are sent should be generated with sufficient
line thickness. Some past figures authors have sent had their line
widths become very faint when SIAM set the papers using a high-quality
1200dpi printer.

Hardcopy for non-{PostScript} figures should be included in the
submission of the hardcopy of the manuscript. Space should be left in
the \texttt{\{figure\}} command for the hardcopy to be inserted in
production.

\section{Bibliography and BibTeX}\label{bibliography-and-bibtex}

If using {Bib}TeX, authors need not submit the {bib} file for their
papers. Merely submit the completed {bbl} file, having used {siam.bst}
as their bibliographic style file. {siam.bst} only works with
BibTeX~version 99i and later. The use of BibTeX~and the preparation of a
{bib} file is described in greater detail in~{[}@Lamport{]}.

If not using BibTeX, SIAM bibliographic references follow the format of
the following examples:

\begin{verbatim}
\bibitem{Ri} {\sc W. Riter},
{\em Title of a paper appearing in a book}, in The Book 
Title, E.~D. One, E.~D. Two, and A.~N. Othereditor, eds., 
Publisher, Location, 1992, pp.~000--000.

\bibitem{AuTh1} {\sc A.~U. Thorone}, {\em Title of paper
with lower case letters}, SIAM J. Abbrev. Correctly, 2
(1992), pp.~000--000.

\bibitem{A1A2} {\sc A.~U. Thorone and A.~U. Thortwo}, {\em
Title of paper appearing in book}, in Book Title: With All
Initial Caps, Publisher, Location, 1992.

\bibitem{A1A22} \sameauthor, % generates the 3 em rule
{\em Title of Book{\rm :} Note Initial Caps and {\rm ROMAN
TYPE} for Punctuation and Acronyms}, Publisher,
Location, pp.~000--000, 1992.

\bibitem{AuTh3} {\sc A.~U. Thorthree}, {\em Title of paper
that's not published yet}, SIAM. J. Abbrev. Correctly, to appear.
\end{verbatim}

Other types of references fall into the same general pattern. See the
sample file or any SIAM journal for other examples. Authors must
correctly format their bibliography to be considered as having used the
macros correctly. An incorrectly formatted bibliography is not only
time-consuming for SIAM to process but it is possible that errors may be
introduced into it by keyboarders/copy editors.

As an alternative to the above style of reference, an alphanumeric code
may be used in place of the number (e.g., {[}AUTh90{]}). The same
commands are used, but \texttt{\textbackslash{}bibitem} takes an
optional argument containing the desired alphanumeric code.

Another alternative is no number, simply the authors' names and the year
of publication following in parentheses. The rest of the format is
identical. The macros do not support this alternative directly, but
modifications to the macro definition are possible if this reference
style is preferred.

\section{Conclusion}\label{conclusion}

Many other style suggestions and tips could be given to help authors but
are beyond the scope of this document. Simple mistakes can be avoided by
increasing your familiarity with how LaTeX~functions. The books referred
to throughout this document are also useful to the author who wants
clear, beautiful typography with minimal mistakes.

\section{The use of appendices}\label{the-use-of-appendices}

The \texttt{\textbackslash{}appendix} command may be used before the
final sections of a paper to designate them as appendices. Once
\texttt{\textbackslash{}appendix} is called, all subsequent sections
will appear as

\section{Title of appendix}\label{title-of-appendix}

Each one will be sequentially lettered instead of numbered. Theorem-like
environments, subsections, and equations will also have the section
number changed to a letter.

If there is only {\emph{one}} appendix, however, the
\texttt{\textbackslash{}Appendix} (with a capital letter) should be used
instead. This produces only the word {\textbf{Appendix}} in the section
title, and does not add a letter. Equation numbers, theorem numbers and
subsections of the appendix will have the letter ``A'' designating the
section number.

If you don't want to title your appendix, and just call it
{\textbf{Appendix A.}} for example, use
\texttt{\textbackslash{}appendix\textbackslash{}section*\{\}} and don't
include anything in the title field. This works opposite to the way
\texttt{\textbackslash{}section*} usually works, by including the
section number, but not using a title.

Appendices should appear before the bibliography section, not after, and
any acknowledgments should be placed after the appendices and before the
bibliography.

{1} , {\emph{The}} LaTeX~{\emph{Companion}}, Addison-Wesley, Reading,
MA, 1994.

, {\emph{Handbook of Writing for the Mathematical Sciences}}, Society
for Industrial and Applied Mathematics, Philadelphia, PA, 1993.

, LaTeX: {\emph{A Document Preparation System}}, Addison-Wesley,
Reading, MA, 1986.

, {\emph{A Beginner's Book of}} TeX, Springer-Verlag, Berlin, New York,
1991.

\end{document}
